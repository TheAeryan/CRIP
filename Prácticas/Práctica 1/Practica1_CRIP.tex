\documentclass{article}

\title{Práctica 1 de Criptografía y Computación}
\date{}
\author{Carlos Núñez Molina}

\usepackage{titlesec}

\titleformat{\section}
  {\normalfont\large}{\thesection}{1em}{}

\begin{document}
	\maketitle
	
	\setcounter{section}{5} % Empezar por el ejercicio 6	
	
	\section{Elige tres números compuestos $n_1$, $n_2$ y $n_3$. El número $n_1$ debe ser un número con tres cifras. Para el número $n_2$ elige 5 primos pequeños (de dos o tres cifras) y multiplícalos. Para el número $n_3$ elige dos primos grandes (con alguna de las funciones que has implementado en los apartados anteriores) y multiplícalos. Para el número $n_1$ calcula todos los falsos testigos. Para $n_2$ y $n_3$ elige una lista al azar de 200 números y calcula cuáles de ellos (y cuántos) son falsos testigos.}
	
	He elegido los números siguientes:
	
	\begin{itemize}
		\item $n_1 = 225$, que es $5^2 * 3^2$.
		\item $n_2 = 123358956461$, que es resultado del producto de los primos					  13, 71, 277, 563 y 857.
		\item $n_3 = 12345679009419752461$, que es resultado del producto de los 			  primos $1000000007$ y  $12345678923$.
	\end{itemize}
	
	El número $n_1$ no tiene ningún falso testigo, es decir, el \emph{Test de Miller-Rabin} devuelve que es un número compuesto para todos los testigos $2<=a<=n_1-2$.
	
	De igual forma, al calcular para los números $n_2$ y $n_3$ cuántos testigos,
	de entre 200 elegidos aleatoriamente, eran falsos, no me salió ninguno.

	\section{Para el número $n = 3215031751$ elige al azar una lista de 200 números (entre 2 y $n-2$) y estudia cuántos son falsos testigos.}
	
	Para el número $3215031751$ he encontrado, de entre 200 testigos elegidos aleatoriamente, a 48 falsos testigos. \newline Los testigos son: 2935071731, 2931603661, 492933966, 1306278621, 1617551744, 31111325, 1498272839, 2929392930, 134849070, 604718046, 1271080353, 518648904, 2706523043, 947697757, 1981132259, 1985011056, 1478515942, 1835796409, 3000274167, 1332276489, 2497397895, 3176671017, 2954931308, 2303475158, 647172536, 3121660915, 2472553653, 2513988608, 1715030066, 2181060077, 1530369159, 3047639753, 1489381660, 40376247, 783840136, 2449633956, 1660703196, 172367108, 1298928945, 2189953363, 2412866465, 544766023, 1920927519, 702355008, 1876283141, 1807417814, 1996235042, 46896364. A pesar de ser un número para el que el Test de Miller-Rabin funciona mal, el número de testigos falsos no supera el máximo de un cuarto de los testigos totales, 50 en este caso.
	
	\section{Para el número $n = 2199733160881$ elige una lista de 100 números (entre 2 y $n-2$) y para cada uno de ellos estudia si es
testigo falso con el test de Fermat y con el test de Miller-Rabin.}	

	Al ejecutar ambos tests con 100 testigos al azar para el número 2199733160881, me salen los siguientes testigos falsos:
	
		\begin{itemize}
		\item \textbf{Test de Fermat} - Todos los testigos (100) son falsos. Los testigos empleados han sido: \newline 1772366524600, 538468577570, 1839695826187, 1830325115900, 245956300488, 252227361614, 883005771162, 2021500875258, 1403506127642, 1058710820099, 1274965998, 2030564265196, 77194776504, 758906814760, 1388808176848, 1916057635589, 165558076252, 981639518090, 2003188658905, 1802896063434, 31718943680, 1755188885177, 55080159044, 708658801812, 1259837072747, 1249683767330, 545458700298, 2023519069048, 1320752862657, 1256519971144, 43764561062, 663683067093, 741443164060, 352702359926, 1010712566517, 1723847715183, 1512087617015, 3464078178, 2030006524057, 1072795148730, 1746683146928, 471553180675, 2092613404374, 332905792210, 1464046351009, 1899649690036, 1287934280612, 336222710000, 265698880005, 1483512052135, 716810812416, 2166646918644, 1296242564689, 811968931578, 996591533843, 516360696245, 1471477762697, 1230483602460, 656229960390, 888344632866, 2148952297126, 103919305770, 720836525341, 1361348230269, 2032161668090, 15772507540, 1831562525323, 108007918840, 47066868762, 1716363657063, 446304106997, 468875340121, 1218827690995, 41944845957, 1068626703798, 1285685482756, 1199452632461, 588446944227, 2146605465064, 76929896121, 877568880872, 715109297197, 1303152843080, 369200433802, 1748369171515, 1724908326282, 653811897833, 1474083710490, 1040767606137, 2131100465512, 1783945736622, 753710302670, 1893649143413, 1511591654224, 507237636126, 1760298245970, 1225874052858, 1070128630683, 378439231928, 472393159198.
		\item \textbf{Test de Miller-Rabin} - Solo hay 9 testigos falsos: \newline 2021500875258, 1058710820099, 1916057635589, 1802896063434, 1249683767330, 446304106997, 468875340121, 1068626703798, 1474083710490.
	\end{itemize}
	
	Esto muestra cómo el Test de Miller-Rabin es mucho más fiable que el de
	Fermat.
	
	\section*{\textbf{Instrucciones de ejecución}}

	Todo el código usado para esta práctica se corresponde con un script de Python denominado \emph{Practica1.py}. Este
	script ha sido programado en Linux usando Python 3.6.
	
	Este archivo no tiene \emph{main} sino que simplemente se compone de las distintas funciones utilizadas en la práctica. Se pueden ejecutar usando el entorno de \textbf{Spyder}. Para ello, simplemente hay que ejecutar una vez el script y, en el 	terminal de iPython que proporciona Spyder, ejecutar
la función deseada.	
	
\end{document}